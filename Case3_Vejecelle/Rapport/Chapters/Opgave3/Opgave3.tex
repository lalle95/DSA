%!TEX root = ../../Main.tex
\graphicspath{{Chapters/Opgave3/}}
%-------------------------------------------------------------------------------

\chapter{System overvejelser}

I denne opgave vil vi undersøge hvor mange betydende cifre vi kan medbringe, hvis vi tager udgangspunkt i 100. ordens filter, og det skal kunne vises på et display. Støjens spredning skal ligge under 1/10 af værdien.
Vi starter med at udregne gram per bit.

\begin{lstlisting}[frame=single]  % Start your code-block
for i:=maxint to 0 do
bits_per_gram=(avg_unloaded-avg_loaded)/1000;
grams_per_bit=1/bits_per_gram;
\end{lstlisting}
Derefter kan vi udregne nøjagtigheden, for vores filter.

\begin{lstlisting}[frame=single]  % Start your code-block
for i:=maxint to 0 do
gram_thresh=dev_unloaded_filtered*grams_per_bit*10;
=151.3545
\end{lstlisting}
Det betyder at vi kan måle med 150 gram præcision, derfor kan vi ikke have flere betydende cifre end heltal. 
