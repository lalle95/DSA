%!TEX root = ../../Main.tex
\graphicspath{{Chapters/Indledning/}}
%-------------------------------------------------------------------------------

\chapter{Indledning}

\section{Baggrund for opgaven}
Mange systemer kan betragtes som målesystemer, hvor man ønsker at måle en bestemt størrelse. Det kunne være vægt, afstand, tryk, temperatur el.lign. Generelt for målesystemer gælder at den værdi, som man måler, vil have en vis usikkerhed. Det er vigtigt at kunne analysere og kvantificere usikkerheden og evt. reducere den vha. midlingsfiltre el.lign.

\section{Formål med opgaven}
Formålet med denne øvelse er, at I skal analysere og forbedre på målesignal fra en vejecelle. Som altid, skal I starte med at analysere signalerne fra sensoren. Dernæst skal I designe og implementere et midlingsfilter (i MATLAB), som kan fjerne støj fra signalet og dermed give mere præcis aflæsning af vægten.
I kan få inspiration fra et realistisk kommercielt vejesystem fra filen ”Weigh\_scale.pdf” - især s. 3-5, som beskriver måling/test på en vejecelle.
